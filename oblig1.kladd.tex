% PREAMBLE

\documentclass{article}
\usepackage[utf8]{inputenc}
\usepackage{hyperref}
\usepackage[T1]{fontenc}

% DOCUMENT
\begin{document}
Obligatorisk oppgave info233, kladd.

\section{Frister, etc.}
Lorem ipsum dolor sic amet.

\section{Om obligen}
I denne obligen skal dere implementere et system som henter ned twittermeldinger, og behandler dem på ulike måter.
Les hele oppgaveteksten før dere setter i gang med oppgaven, og tenk igjennom hva dere vil gjøre og hvordan før dere begynner å skrive.
Det vil spare dere for tid og frustrasjon.
Vi setter som krav at dere skriver enhetstester (I JUnit) for koden deres.
For å hjelpe dere litt i gang med tenkingen har vi gitt dere noen grensesnitt (interfjes) som beskriver metodene dere skal implementere.
Hvordan dere implementerer disse, er helt opp til dere.

\section{Obligen}

Dette prosjektet omhandler et system for nedlasting av tweets, dvs. meldinger fra Twitter. Dere skal først definere datatyper som behøvs. td. TwitterMelding, TwitterBruker, osv.
Du må også definere enhetstester (JUnit) for klassene dine, for alle relevante metoder. Husk equals og evt. compareTo dersom du skal sortere.
Sett deg inn i hvordan du skal bruke koden som vi har gitt deg, det sparer deg for problemer i lengden.

 
Du skal laste ned tweets fra et bestemt sted til en liste (\href{http://docs.oracle.com/javase/7/docs/api/java/util/List.html}{java.util.List}, ikke den i awt) og returnere denne. Dere skal ha en main metode som tester denne klassen ved å instanisere den, laste ned en liste, og skrive den ut.
Husk å definere JUnit tester for å sjekke at alt er korrekt.
 
Du skal lage et grafisk grensesnitt for å vise tweetene.
Vi godtar enkle løsninger, men det må være ryddig.
 
Lagring, dvs. serialisering og deserialisering.
Du skal lagre og lese opp data fra disk. For enhetstesting kan \href{http://docs.oracle.com/javase/7/docs/api/java/io/File.html#deleteOnExit\%28\%29}{File\#deleteOnExit()} være relevant.
    
Filtrering og framvisning: Du skal la brukeren sortere tweets etter tid (stigende og synkende), og vise tweets fra bare en bestemt bruker.

% Trikset her blir vel compareTo, samt filtermetoder og predikater?
% for(TwitterTweet tt : tweets){
%  if(predikat(tt)){
%    samling.add(tt);
%  }
%}
 
Statistikk på forfattere:
 Antall tweets
 Totalt tegn i tweets
 Gjennomsnittslig antall tegn per tweet. % Husk heltallsdivisjon

Fiks ÆØÅ
 Det ser ikke ut, og vi vil at det skal se ut.
 Dette må fikses.
 
\subsection{Bonuspoeng}
Ideer:

Ordfrekvens
Tweets med banneord?

\section{Tips og triks}
\subsection{gson biblioteket}
Tweetene dere får er i JSON-formatet, \url{http://www.json.org/} og dere må selv hente data ut fra disse dataene.
For å hjelpe dere med dette har vi lagt ved Jackson, et bibliotek for å behandle JSON-data.

Vi har også med en demonstrasjon av bruken av dette biblioteket i prosjektet.
Dersom dere lurer på hvordan en skal lese inn data fra JSON, så er demoen i aller høyeste grad relevant.

\end{document}
