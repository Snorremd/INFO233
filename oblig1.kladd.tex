% PREAMBLE

\documentclass{article}
\usepackage[utf8]{inputenc}
\usepackage{hyperref}
\usepackage[T1]{fontenc}

% DOCUMENT
\begin{document}
Obligatorisk oppgave info233, kladd.

\section{Frister, etc.}
Utlevert [dato], frist [data] klokken [klokkeslett]
Denne obligatoriske oppgaven vurderes til godkjent/ikke godkjent.
De som ikke ikke leverer, etc.
De som får ikke godkjent - ? (Ny endelig frist til å rette?)

Leveringsformatet er et Eclipseprosjekt, eksportert til en komprimert fil. (zip eller tar.gz?)

\section{Oppgaven}

Obligatorisk Oppgave 1 omhandler et system for prosessering av Tweets (dvs. Twitter-meldinger) og Twitter-brukere fra en Twitter-strøm lokalisert til Sør-Norge. I oppgaven skal dere først definere datatyper som behøves, for eksempel TwitterMelding, TwitterBruker osv. Dere må også definere enhetstester (JUnit) for klassene og de operasjonene (metodene) de skal tilby. Dersom det er behov for sortering eller likhetssjekker for objektene av en datatype må dere huske compareTo og equals-metodene.


I forbindelse med den obligatoriske oppgaven vil dere få utdelt et delvis implementert prosjekt som består av tre deler:
\begin{itemize}
  \item En klasse TwitterStream som håndterer kommunikasjon med Twitter-strømmen og lar en hente ut Tweets.
  \item En rekke Java Interface som fungerer som kravspesifikasjon for de datatypene og operasjonene som skal implementeres.
  \item Eksempler (demo-pakken) som viser hvordan man bruker Jackson json-parseren, Apache commons lang for tekstbehandling og hvordan man kan bruke TwitterStream-klassen for å hente ut Tweets fra twitter-strømmen.
\end{itemize}

\subsection{Twitter-strømmen}
TwitterStream-klassen implementerer en enkel Singleton-klasse for å kommunisere med Twitter-strømmen. Bruken av denne klassen er illustrert i demo-pakken i Java-prosjektet. Klassen har fire metoder. En statisk metode instance() for å hente en instans av klassen (påkaller konstruktøren), tre instans-metoder hhv. connect() for å koble til strømmen, disconnect() for å koble fra, og getTweets(int number) for å hente ut Twitter-meldinger.

Det er nødvendig å spesifisere en twitterauth.properties-fil med autentiseringsdata i config-pakke for å få koblet seg på Twitter-strømmen. For å få tak i autentiseringsdata må dere logge på med en Twitter-konto på https://dev.twitter.com. Gå deretter til My Applications-linken under profilbildet og velg Create new Application og fyll ut info. Etter at du har opprettet applikasjonen kan du trykke på Create access token-knappen for å generere access token-nøkler. Disse tilsvarer token og secret-verdiene i twitterauth.properties-filen.

Hvis dere opplever problemer med autentisering av applikasjonen kan dere få hjelp av studentassistentene.

\section{Tips og triks}
Tweetene dere får er i \href{http://www.json.org/}{JSON-formatet}, og dere må selv hente data ut fra disse dataene.
For å hjelpe dere med dette har vi lagt ved Jackson, et bibliotek for å behandle JSON-data.

Vi har også med en demonstrasjon av bruken av dette biblioteket i prosjektet.
Dersom dere lurer på hvordan en skal lese inn data fra JSON, så er demoen i aller høyeste grad relevant.

StackOverflow og Java sine API sider er ellers særdeles nyttige.
Til slutt foreslår vi at dere kommer på gruppene og spør spørsmål på ting dere sitter fast på.

\subsection{Kravspesifikasjon}
Interfacene spesifiserer et minimumskrav til hvilke datatyper og operasjoner som må være tilgjengelige. Dere står selvsagt fritt til å implementere flere metoder, hjelpermetoder, klasser osv. Hvordan dere implementerer klassene er opp til dere. De eneste kravene som stilles er at:
\begin{enumerate}
\item Dere har en GUI som er oversiktlig og brukendes
\item Alle metodene som er spesifisert er korrekt.
\item At dere kan lagre og åpne tidligere økter. (vha. serialisering)
\item At dere har skrevet tester for å sjekke oppførsel (i JUnit).
\item At koden er korrekt.
\end{enumerate}

På de neste sidene følger en oversikt over interfacene som er lagt ved i kravspesifikasjon-pakken.


\newpage

\subsubsection{TwitterMelding}
\label{subsec:TwitterMelding}

\begin{description}
\item [getMeldingsTekst():String] \ \\ % Det må være en bedre måte å gjøre dette på.
  Meldingsteksten som følger med tweeten.
  Det er ikke nok å bare ta med teksten, du må også ta hånd om unicode escape sekvensene. Til dømes er \textbackslash u2013 en annen måte å skrive ``–'' på.
\item [getBruker():TwitterBruker] \ \\
  Skal returnere brukeren som sendte meldingen.
  Alle meldinger skal ha en bruker
\item [size():int] \ \\
  Returnerer størrelsen på meldingen, per antall tegn.
\item [dato():Calendar] \ \\
  Returnerer når meldingen ble sendt, representert som et \href{http://docs.oracle.com/javase/7/docs/api/java/util/Calendar.html}{Calendar} objekt.
\item [getID():String] \ \\
  Returnerer den unike identiteten til meldingen.
\end{description}
\newpage
\subsection{TwitterMeldingCollection}
\label{subsec:TwitterMeldingCollection}
\begin{description}
\item [size():int] \ \\
  Størrelsen på samlingen
\item [insert(TwitterMelding element):boolean] \ \\
  Setter inn en melding et sted i samlingen. boolskPLH
\item [insert(TwitterMelding element, int index):boolean] \ \\
  Setter inn en melding på angitt sted i samlingen. boolskPLH
\item [remove(int index):boolean] \ \\
  Fjerner et element på angitt sted i samlingen. boolskPLH
\item [get(int index):TwitterMelding] \ \\
  Henter ut en TwitterMelding på angitt sted i samlingen.
\item [getTweetsWith(String word):Collection<TwitterMelding>] \ \\
  Henter ut TwitterMeldinger med et angitt ord i en samling.
\item [deleteTweet(TwitterMelding tm):boolean] \ \\
  Sletter angitt melding fra samlingen. boolskPLH
\item [getTweet(String id):TwitterMelding] \ \\
  Returnerer en melding med gitt ID
\item [deleteTweet(String id):boolean] \ \\
  Sletter en melding med gitt ID. boolskPLH
\item [sortertEtterTid(boolean nyesteFørst):Collection<TwitterMelding>] \ \\
  Gir ut en samling av TwitterMeldinger som er sortert etter tid.
  Dersom nyesteFørst er sann, skal de nyeste fremst i samlingen, ellers skal de eldste fremst.
\item [sortertEtterLengde(boolean lengsteFørst):Collection<TwitterMelding>] \ \\
  Gir ut en samling av TwitterMeldinger som er sortert etter lengde.
  Dersom lengsteFørst er sann, skal de lengste fremst i samlingen, ellers skal de korteste først.
\item [meldingerFra(TwitterBruker bruker):TwitterMeldingCollection] \ \\
  Gir ut en TwitterMeldingCollection med alle meldingene fra den angitte brukeren.
  Rekkefølgen spiller ingen rolle.
\end{description}
\newpage
\subsubsection{TwitterBruker}
\label{subsec:TwitterBruker}
\begin{description}
\item [getNavn():String] \ \\
  Returnerer navnet på brukeren
\item [getID():String] \ \\
  Returnerer den unike IDen til brukeren.
\item [numTweets():int] \ \\
  Returnerer antall meldinger brukeren har sendt
\item [setTweets(int numTweets):void] \ \\
  Setter antall meldinger denne brukeren har sendt
\item [numCharacters():int] \ \\
  Henter ut det totale antall tegn i alle meldinger denne brukeren har sendt
\item [getAverageNumCharacters():double] \ \\
  Henter ut den gjennomsnittslige lengden på en melding fra en bruker.
\item [numFollowers():int] \ \\
  Returnerer antall followers denne brukeren har
\item [setFollowers(int followers):void] \ \\
  Setter antall followers denne brukeren har
\item [numFriends():int] \ \\
  Returnerer antall friends denne brukeren har
\item [setFriends(int friends):void] \ \\
  Setter antall friends denne brukeren har
\end{description}
\newpage
\subsubsection{TwitterBrukerCollection}
\label{subsec:TwitterBrukerCollection}

\begin{description}
\item [size():int] \ \\
  Størrelsen på samlingen
\item [insert(TwitterBruker tb):boolean] \ \\
  Setter inn en bruker et sted i samlingen. boolskPLH
\item [insert(E element, int index):boolean] \ \\
  Setter inn en bruker på angitt sted i samlingen. boolskPLH
\item [remove(int index):boolean] \ \\
  Fjerner en bruker på angitt plass. boolskPLH
\item [get(int index):TwitterBruker] \ \\
  Henter en bruker fra angitt plass. boolskPLH
\item [getBruker(String navn):TwitterBruker] \ \\
  Henter ut en bruker fra samlingen
\item [sortertEtterMeldinger(boolean stigende):Collection<TwitterBruker>] \ \\
  Henter ut en samling av brukere, sortert etter antall meldinger.
  Hvis stigende er sann, sorter i stigende rekkefølge, ellers i synkende rekkefølge.
\item [sortertEtterAntallFollowers(boolean stigende):Collection<TwitterBruker>] \ \\
  Henter ut en samling av brukere, sortert etter antall followers.
  Hvis stigende er sann, sorter i stigende rekkefølge, ellers i synkende rekkefølge.
\item [sortertEtterAntallFriends(boolean stigende):Collection<TwitterBruker>] \ \\
  Henter ut en samling brukere, sortert etter antall friends.
  Hvis stigende er sann, sorter i stigende rekkefølge, ellers i synkende rekkefølge.
\end{description}

%Gamle teksten.
% Trikset her blir vel compareTo, samt filtermetoder og predikater?
% for(TwitterTweet tt : tweets){
%  if(predikat(tt)){
%    samling.add(tt);
%  }
%}
\end{document}
